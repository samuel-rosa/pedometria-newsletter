\title{Carta ao editor}
\subtitle{Réplica à Igo Lepsch sobre ``Mapeamento de Solos no nosso tempo''}
\author{por Marcos Ceddia}
\maketitle

\begin{wrapfigure}{l}{0.15\textwidth}
\includegraphics[width=0.15\textwidth]{figuras/marcos}
\end{wrapfigure}

Prezado Igo,

Obrigado pelas suas considerações. Embora o Sr. tenha comentado que sua pergunta talvez fosse ``um detalhe semântico, de menor importância'', dou, ao contrário, todo o valor à questão semântica. Se observarmos na literatura, muitas vezes detalhes semânticos geram confusões infindáveis que somente servem para criar mal entendidos de consequências inimagináveis. Por exemplo, por questões semânticas e interpretações das mais variadas, textos religiosos criaram barreiras intransponíveis entre seres humanos. Para Sócrates a ignorância é a origem do mal. De alguma forma, uma palavra mal colocada e mal interpretada contribui para a ignorância. No meio acadêmico também temos vários problemas, tais como: ensino equivocado, proliferação de versões não autorizadas, desvio da razão pelas quais se faz pesquisa, entre outras. Mais especificamente em Ciência do Solo, veja o problema que se criou quando o Mapeamento Digital de Solos iniciou!? Quanto problema gerou e ainda gera o termo ``Digital''? Por isso, não considero a discussão semântica uma perda de tempo ou de menor importância. Na verdade, sua pergunta abre uma grande oportunidade para rever conceitos, lembrar um pouco da história do levantamento e taxonomia de solos e, sobretudo, para desenvolver uma discussão contextualizada de questões tais como: Mapear classes ou atributos? Existe coerência entre a natureza do que mapeamos e a representação cartográfica dos mapas?

Assim, começamos pelos conceitos de Unidades de Mapeamento e Unidades Taxonômicas, afinal, os levantamentos pedológicos são executados com o apoio de sistemas taxonômicos de classificação.

\subsubsection{Unidade Taxonômica}

De acordo com os \textit{Procedimentos Normativos de Levantamentos Pedológicos} (EMBRAPA, 1995), uma Unidade Taxonômica é uma classe de solo definida e conceituada, segundo parâmetros de classificação. A conceituação é feita segundo um conjunto de características e propriedades do solo, conhecidas por meio de pedons e polipedons (uma entidade espacial). Uma unidade taxonômica corresponde à unidade de classificação mais homogênea em qualquer nível categórico de sistemas taxonômicos.

Aqui cabem algumas considerações. Como o senhor bem pontuou, uma Unidade Taxonômica é uma concepção teórica e é criada para facilitar o conhecimento sobre solos. Além disso, é uma concepção teórica passível de mudança, mais ou menos frequente, dependendo do sistema taxonômico que adotamos. O Sistema Brasileiro de Classificação de Solos (SiBCS), por exemplo, muda com muito mais frequência do que o sistema americano (Soil Taxonomy).

\subsubsection{Unidade de Mapeamento}

Unidades de mapeamento são áreas de solos definidas em função das unidades taxonômicas que as compõem. Trata-se de um conjunto de áreas de solos com relações e posições definidas na paisagem. Uma unidade de mapeamento é estabelecida e definida para possibilitar a representação cartográfica e mostrar a distribuição espacial de unidades taxonômicas.

Ao lermos o conceito de unidade de mapeamento (objetos reais que mapeamos e representamos no mapa), percebemos o quão estreita é a relação entre unidade taxonômica e unidade de mapeamento. Sem a unidade taxonômica, não existe a unidade de mapeamento com a nomenclatura que usamos, a qual é em essência uma concepção teórica. Uma perturbação relativamente frequente, a qual é consequência dessa estreita relação, é que comumente precisamos reclassificar os nomes que damos as unidades de mapeamento de cartas de solos.

Ainda, a relação entre unidade taxonômica e de mapeamento, recomento aos leitores o texto de Scull et al. (2003) intitulado \href{http://ppg.sagepub.com/cgi/content/abstract/27/2/171}{\textit{Predictive soil mapping: a review}}. Nesse trabalho, além de outros tópicos, o autor relembra a origem do sistema taxonômico de solos e o mecanismo utilizado para adequar a natureza altamente variante do solos e seus atributos com as limitações cartográficas da época. De acordo com Scull et al. (2003) o mapeamento de solos tem sido e continua sendo diretamente influenciado pelo processo de classificação do solo. O Autor cita o caso do Soil Taxonomy, o qual foi influenciado pela taxonomia biológica do século 19 e a prática do levantamento geológico. Aqui apresento a tradução exata do texto de Scull et al. (2003), onde fica bastante claro a estreita ligação entre unidades taxonômicas e unidades de mapeamento (trecho do segundo parágrafo da página 175).}

\begin{quotation}
  ``A proposta do Soil Taxonomy foi fornecer uma maneira objetiva para classificar sistematicamente o solo e foi adotado em uma época em que a informação do solo tinha que ser abstraída para o nível de perfil modal (classificado no Soil Taxonomy), porque era impossível catalogar e apresentar totalmente a variabilidade do solo. Com o objetivo de mapear as unidades taxonômicas, o solo tinha que ser distinguido como uma unidade espacial, um pedon (a menor unidade reconhecível que pode ser chamado de solo). Na prática, esta distinção espacial do solo resultou em um mapa onde as classes são unidades homogêneas com variabilidade desconhecida e limites pontuais (mapas coropléticos).''
\end{quotation}

Nesse ponto começamos a notar que também a Unidade de Mapeamento não pode ser considerada tão real assim, pois esta é na verdade uma maneira que os pedólogos criaram para representar espacialmente aquilo que se descreveu no campo e recebeu um nome teórico (nome da unidade taxonômica). Ou seja, sabe-se que o perfil modal não é homogêneo ao longo da paisagem (sua variação é contínua), mas o pedólogo (após explorar a área) elege alguns perfis modais para representar no mapa a distribuição espacial dos solos da área de estudo. A unidade de mapeamento passa a impressão de que consiste em um território homogêneo com limites abruptos. Sabemos que isso não é uma verdade, mas aceitamos como um ``modelo'' de representação espacial da distribuição dos solos (a carta de solos). Claramente, percebe-se que os mapas coropléticos não representam adequadamente a distribuição espacial da real variabilidade espacial dos solos. Importante notar que esse procedimento de mapeamento e representação cartográfica refletiu a disponibilidade tecnológica da época.

A partir do acima exposto também fica claro uma clara distinção entre o Mapeamento Digital de Solos e o Tradicional. Como a pesquisa em mapeamento digital de solo tem desenvolvido métodos de levantamento de solos que descrevem mais acuradamente a relação solo paisagem, comumente, tais métodos estão em desacordo com a abordagem tradicional. Os métodos digitais, frequentemente não adotam o conceito de solo como uma entidade espacial. Os métodos digitais estão mais focados no mapeamento da variabilidade espacial e temporal dos atributos dos solos (teor e estoque de carbono, retenção e disponibilidade de água, contaminantes e etc) e isso tem sido dificultado devido à defesa do Soil Taxonomy (Scull et al. 2003).

Após as considerações acima, agora retorno à sua pergunta referente ao termo que utilizei e que resultou em sua pergunta: ``Um modelo da distribuição espacial de classes e/ou atributo de solos em uma determinada área de interesse''. Na verdade o termo classes (o qual não fui eu o inventor) é utilizado para informar que o mapa (digital ou não) distingue espacialmente unidades de mapeamento que recebem o nome de uma unidade taxonômica. O modelo, nesse caso, representa a variabilidade espacial do solo com os ``defeitos'' da representação na forma de mapas coropléticos (percepção de variabilidade homogênea e com limites abruptos). Aqui podemos aproveitar para apresentar duas concepções de representação espacial que também geram polêmica, denominadas \textit{visão de objetos} e \textit{visão de campo}. Os mapas coropléticos seguem a visão de objetos, na qual as unidades de mapeamento (os objetos) são entidades independente no espaço. Na visão de campo (formato raster) a representação do mapa é feita através de uma matriz regular de linhas e colunas. Cada célula ou pixel recebe um valor (valor do atributo do solo ou classe de solo) e representa uma determinada área em função do tamanho de cada pixel (resolução). A visão de campo representa melhor o caráter contínuo da variabilidade espacial do solo. Além disso, é melhor para fazer análises espaciais em SIGs.

\subsubsection{Conclusões e considerações}

Podemos assim resumir que ao fazermos um mapa de classes de solos estamos representando o solo de forma irreal (a unidade de mapeamento não reflete a variabilidade espacial do solo) e também dando um nome que nada mais é que uma concepção teórica.

Minha humilde opinião é que os pesquisadores devem refletir melhor sobre a exata dimensão da taxonomia de solo sob o aspecto de mapeamento de solos. É inegável que com as tecnologias e ferramentas computacionais modernas obteve-se um ganho expressivo para se fazer levantamento e predição de atributos dos solos. Além disso, em muitos casos, o conhecimento da drenagem, disponibilidade de água, cor, teor de elementos químicos é de interpretação mais direta para um usuário do que dizer que uma parte do território á denominado por um nome taxonômico puramente teórico e voltado para fins acadêmico-científicos. Para um usuário de mapas de solos, uma mapa de classes de solos (tradicional ou digital) pode sim ser visto como um mapa abstrato.

\address{Marcos B. Ceddia\\
  \begin{footnotesize}
    Universidade Federal Rural do Rio de Janeiro\\
    \email{marcosceddia@gmail.com}
  \end{footnotesize}
}
%%% Local Variables:
%%% mode: latex
%%% TeX-master: 3rd-edition.tex
%%% End: