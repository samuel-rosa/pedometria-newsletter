\title{Carta ao editor}
\subtitle{Tréplica à Marcos Ceddia sobre ``Mapeamento de Solos no nosso tempo''}
\author{por Igo Lepsch}
\maketitle

\begin{wrapfigure}{l}{0.15\textwidth}
\includegraphics[width=0.15\textwidth]{figuras/igo}
\end{wrapfigure}

Caro Marcos,

Interessante as suas considerações. Quero completá-las com algumas citações e opiniões.

Nossa conversa começou com uma discussão sobre ``mapas de classes de solo'' e ``mapas de atributos do solo''. Você cita que Scull et al. (2003) é de opinião que o mapeamento de solos tem sido e continua sendo diretamente influenciado pelo processo de classificação do solo. Concordo com esses autores mas penso que um mapeamento de solos não deve ser influenciado por sistemas de classificação de solos.

Creio que os - por alguns autores - denominados ``mapas de classes de solos'' são na verdade mapas de indivíduos de solos ou simplesmente ``mapas de solos'' já que, em pedologia, solos são definidos como ``corpos naturais''.

Copiando algumas frases do livro de Jean-Paul Legros (\textit{Mapping of the Soil, 2006}):

\begin{quotation}
  ``Viewing soils as distinct individuals is very efficient from several points of view. To begin with, it allows us to structure our knowledge in the form of classifications. Then, it makes possible the drawing of maps on which boundaries appear. Thirdly, the entities thus shown can be processed by means of computer systems suitable for the creation of categorical maps. But some authors judge this attitude rather harshly: \textit{`The survey and classification of geology, soil or vegetation into exact, sharply defined} classes \textit{or areas that meet abstract ideas of nicely defined, properly circumscribed} units \textit{has been an enormous exercise in forcing continuously varying phenomena into exact moulds. Although much scientific evidence has been assembled to demonstrate that} soils \textit{are not exact, homogeneous entities[...]'''}(Burrough and Franck, 1995).
\end{quotation}

Legros (2006) afirma que nem todos autores concordam com o corpo de solo tal como maioria dos pedólogos o define. Contudo, a maior parte dos pedólogos, que durante muitos anos mapearam - em detalhe - solos no campo usando conhecimentos tácitos, acreditam realmente que ''indivíduos solo'' existam; e eles os tem delineado em muitos mapas de muitas regiões. Em respeito a esses colegas consideremos que os corpos de solo existam mesmo e que os delineamentos que fazemos nos mapas e que mais se aproximam destes corpo natural sejam as séries de solo.

Temos que considerar também que um trabalho, que visa à compreensão e modelagem da natureza, só será social e economicamente justificado se os relatórios correspondentes puderem ser aplicados (ou seja, se incluírem interpretações para usos práticos).

Consideremos agora que, para que um mapa de solos tenha aplicação prática ele deve ser feito numa escala que possa ser útil aos usuários; isto é, deve ser um levantamento detalhado, definido como aquele que deve ser executado ``em nível de séries''. Há que ponderar também que os relatórios desses levantamentos, para terem seu custo justificado, devem incluir interpretações para agricultura, engenharia civil, etc (tal como os relatórios dos ``soil surveys'' dos EUA).

No Brasil - ao contrário de muitos outros países (como EUA e os nosso vizinhos argentinos) - série de solos ainda não foram definidas, raros são os levantamentos detalhados executados por órgãos do Governo e raros também são os relatórios que apresentam aplicações para usos práticos.

O país que mais executou levantamentos detalhados e mais os usa para fins práticos é os EUA. Em 1960 boa parte do território americano já estava mapeada (escalas próximas de 1:30.000) com mais de dez mil séries identificadas. E esses mapas não eram de classes de solos, uma vez que séries de solos só foram consideradas como classes após o aparecimento do ``Soil Taxonomy''. Mesmo assim, houve muitas dúvidas se séries de solos deviam ser consideradas como classes ou não. Abaixo algumas frases copiadas de uma \href{http://www.nrcs.usda.gov/wps/portal/nrcs/detail/soils/planners/?cid=nrcs142p2_053571}{nota técnica do USDA-NRCS} que informa sobre isso.

\begin{quotation}
  ``Today we readily accept that a soil series is the lowest level of Soil Taxonomy and its properties can not extend beyond the limits of the family to which it belongs. This was not always so easily accepted. The 1965 NCSS conference proceedings includes a discussion of \textit{guidelines for allowable tolerances in the stretching of family class limits by series class limits}. The debate was whether the range in characteristics for a series must be within the limits of the family, or alternatively, should only the typical pedon itself have to fit within the family while its range could extend beyond the family limits. The first alternative was agreed to.
    
  The decision to restrict series ranges to family limits has presented some difficulty ever since. We know that natural soil bodies commonly have ranges in properties that straddle one or more of the rigid taxonomic class boundaries. Over the years we have used devices such as taxadjuncts, variants, and \textit{similar soils} to reconcile observations from natural bodies of soils with taxonomic limits. These concepts are awkward and not well understood beyond the soil survey community. In 1977 Dr. Marlin Cline wrote \textit{At the lowest level of the system, we will have to acknowledge the differences between taxonomic soil series and mapping units that bear the same name and will probably have to rectify the confusion this causes. It is conceivable that soil families could become the lowest category of taxonomy, but some ingenious person may find a better solution.}''
\end{quotation}

E, mais adiante:

\begin{quotation}
  ``The distinction between taxonomic classes and components of map units needs to be understood. \textbf{We do not map taxonomic classes} [grifo meu]. We use conceptual landscape models to map natural bodies of soils. We then use our taxonomy to classify and name the soils we have mapped.''   
\end{quotation}

Portanto, creio que: se o mapeamento de solos tem sido e continua sendo diretamente influenciado pelo processo de classificação do solo e se o ``Soil Taxonomy'' é a classificação mais usada, creio que é porque está havendo confusão entre Unidade de Mapeamento (que são objetos) e Unidade Taxonômica (que são conceitos abstratos). Vale a pena ler algo sobre isso no \href{http://casoilresource.lawr.ucdavis.edu/w/images/d/d8/CHAPTER_5_Application_of_Soil_Taxonomy_to_Soil_Surveys.pdf}{capítulo 5} da Nova Edição do ``Soil Taxonomy'':

\begin{quotation}
  ``Soils are landscapes as well as profiles (USDA, SCS, 1951, pp. 5-8; USDA, SCS, 1993, pp. 9-11). In soil survey, a soil-landscape unit can be thought of as a landscape unit (landscape, landform, or landform component) further modified by one or more of the soil-forming factors. Within a soil-landscape unit, the five factors of soil formation interact in a distinctive manner. As a result, areas of a soil-landscape unit have a relatively homogeneous soil pattern. A soil surveyor perceives soil patterns by first conceptually dividing the landscape into soil-landscape units. The boundaries between dissimilar soil-landscape units are placed where one or more of the soil-forming factors change within a short lateral distance.''
\end{quotation}

Convém informar que o Comitê executivo do Sistema Brasileiro de Classificação de Solos (SiBCS) está revendo a conceituação de séries como Unidade Taxonômica. Muitos países que já tem suas séries de solo definidas mas, não as consideram como classes. Provavelmente as primeiras séries de solo brasileiras não serão classes (ou táxons) tal como não o foram as milhares primeiras séries dos EUA. Afinal, para que uma série se solo seja ``batizada'' é necessário que primeiro ela tenha uma área mínima mapeada (normalmente 1.000 ha). Se assim for, e considerando como correta a expressão da nota técnica número quatro do USDA-NRCS de que ``[...] we do not map taxonomic classes[...], we use conceptual landscape models to map natural bodies of soils[...]'', como primeiro conceituar ``séries classes'' sem antes termos ``séries unidade de mapeamento''? Como definir conceitos sem termos os objetos?

Com respeito a suas observações sobre caracterização das unidades de mapeamento, que hoje muitos fazem o que chamam ``perfil representativo'', isso é uma preocupação de longa data. Sobre isso veja alguns comentários de S. W. Buol em \href{http://base.dnsgb.com.ua/files/book/Agriculture/Soil/Soil-Classification.pdf}{``Philosophies of Soil Classifications: From is to does''}:

\begin{quotation}
  ``Cursorily observation of a soil reveals a vertical arrangement of soil components that change, often gradually, as one traverses the landscape. Our understanding of soil is limited without utilization of chemical, mineralogical, biological and physical quantification of soil samples. Soil can be dismembered, sampled and autopsied but only if we know from where each soil sample is located within a body of soil do the analyses help us understand both what a soil is and what a soil does. Identification of depth and vertical arrangement of a soils component parts via several degrees of sophistication i.e. topsoil, subsoil, generic horizon nomenclature and diagnostic horizons is endemic to all attempts to classify soil.

  Perhaps no single problem has plagued soil classification more than identification of the spatial boundaries of a soil individual on the landscape that is to be sampled for study and used as a unit of classification. As soil science struggled to claim adulthood as an independent science a basis of identifying a soil individual suitable for classification has been at the root of many philosophies. Soil classification is closely allied with soil mapping and most practitioners intertwined pragmatic realities of mapping into their philosophies of identifying an appropriate volume of soil for classification''.
\end{quotation}

Creio que a pedometria pode nos ajudar muito a identificar o melhor local para amostrar os ``perfis modais'' depois que as unidades de mapeamento de solos forem delineadas com base em modelos conceituais solo-paisagem. Perfis estes que realmente represente nosso objeto de estudo: os corpos naturais dos solos. Por vezes, tenho a impressão que muito do que chamamos ``perfis representativos'' são escolhidos mais para representar conceitos (classes definidas no SiBCS) do que os objetos que mapeamos. Sobre essa problemática veja o que afirmamos \href{http://www.e-publicacoes.uerj.br/index.php/tamoios/article/view/5665/5196}{em um artigo nosso}:

\begin{quotation}
  ``Com a ênfase em características dos perfis do solo muitos pedólogos brasileiros tendem a ter uma ``visão afunilada'' das terras considerando muito mais os perfis expostos em trincheiras abertas em um único local do corpo do solo sem tentar integrá-lo na paisagem como um todo. Em outras palavras a atenção maior tem sido na IMAGEM (perfil do solo) e CONCEITO (classificação) deixando o OBJETO (corpo de solo) em segundo plano.''
\end{quotation}

\address{Igo F. Lepsch\\
  \begin{footnotesize}
    Pesq. Visistante; ICA-UFMG, Montes Claros, MG\\
    \email{igo.lepsch@yahoo.com.br}
  \end{footnotesize}
}
%%% Local Variables:
%%% mode: latex
%%% TeX-master: 3rd-edition.tex
%%% End:
