\title{Carta ao editor}
\subtitle{Questionamento à Marcos Ceddia sobre ``Mapeamento de Solos no nosso tempo''}
\author{por Igo Fernando Lepsch}
\maketitle

\begin{wrapfigure}{l}{0.15\textwidth}
\includegraphics[width=0.15\textwidth]{figuras/igo.pdf}
\end{wrapfigure}

Caro Marcos Bacis Ceddia,

Gostei muito de seu artigo ``Mapeamento de Solos no nosso tempo'', publicado no número 2 do ``Pedometria Newsletter/SBCS'', no qual destaca a importância do uso de técnicas modernas nos levantamentos de solos.

No texto há a seguinte afirmação:

\begin{quotation}
  ``Sobre a ótica de modelos um mapa de solos pode ser entendido como um \textit{modelo da distribuição espacial de classes e/ou atributo de solos em uma determinada área de interesse}.''

\end{quotation}

É apenas um detalhe semântico, de menor importância, mas gostaria que ponderasse e nos respondesse, uma vez que tenho notado, em muitos trabalhos, alguma confusão entre Unidades de Mapeamento de Solo (que representam objetos reais) e Unidades Taxonômicas (representadas pelas classes):
Como em levantamentos pedológicos devemos considerar que o que mapeamos são objetos reais (os corpos de solos) e considerando que classes de solos são conceituais (isto é, abstrações, sem existência real) você considera correta a expressão ``distribuição espacial de classes de solos'' que usou em seu texto uma vez ser impossível ``mapear abstrações''?

Abraços.

\address{Igo F. Lepsch\\
  \begin{footnotesize}
    Pesq. Visistante; ICA-UFMG, Montes Claros, MG\\
    \email{igo.lepsch@yahoo.com.br}
  \end{footnotesize}
}
%%% Local Variables:
%%% mode: latex
%%% TeX-master: 3rd-edition.tex
%%% End: