\title{Editorial}
\maketitle

É com prazer que publicamos a terceira edição de \pedometria. Iniciamos um novo ano com energias renovadas para dar continuidade à tarefa de tornar a pedometria um tema familiar à comunidade acadêmica brasileira.

Na edição passada tivemos importantes contribuições que fomentaram discussões conceituais, não apenas da pedometria, mas da ciência do solo como um todo. Tais discussões iniciaram com o Congresso Brasileiro de Ciência do Solo realizado em Florianópolis onde, pela primeira vez na história da ciência do solo brasileira, a pedometria teve espaço próprio, contando com a presença de inúmeros pesquisadores de outros países, como Alex McBratney, Sabine Grunwald e David Rossiter, sem falar nas brilhantes apresentações de pesquisadores brasileiros como Marcos Ceddia e Maria de Lourdes. Os reflexos destas discussões aparecem na presente edição por meio de contribuições de Igo Lepsch e Marcos Ceddia, que dialogam sobre a fundamentação teórica do mapeamento de solos. Esperamos que esse diálogo motive alguns de nossos inúmeros leitores, como Élvio Giasson, Márcio Francelino, Lúcia Anjos, Ricardo Dalmolin, Gustavo Vasques, César Chagas, Alexandre ten Caten, Carlos Alberto Flores, entre outros (são tantos leitores que não há espaço para listar todos, a contribuir na próxima edição de \pedometria{} apresentando seus pontos de vista sobre o tema.

Também apresentamos mais um colega docente e pesquisador em pedometria, Ricardo Bergamo Schenato, da Universidade Federal de Santa Maria. Ricardo possui larga experiência em modelagem dos ciclos do carbono e nitrogênio no solo e seu efeito na emissão de gases de efeito estufa, e acredita o ensino de informática, matemática e estatística precisa ser reforçado nos cursos de graduação. Diríamos que se trata de um grande entusiasta!

Por fim, um belo toque de interdisciplinaridade é dado por Diego Siqueira e José Marques Jr. ao discutirem a relação entre a cela unitária dos minerais, orientação atômica, magnetismo, formação do solo, sensoriamento remoto e mapeamento digital de solos.

Na próxima edição devemos ter uma contribuição de Alexandre Demattê e sua equipe sobre a Biblioteca Espectral de Solos do Brasil (BESB) que, devido ao intenso trabalho demandado pelo belíssimo projeto sendo conduzido na Esalq, não ficou pronto a tempo para a atual edição de \pedometria. Novidades também deverão aparecer na próxima edição de \pedometria, como a inclusão de novos membros na comissão editorial e a adoção de uma plataforma online para elaboração e edição dos artigos.

Deliciem-se com a terceira edição de \pedometria!

%%% Local Variables:
%%% mode: latex
%%% TeX-master: 3rd-edition.tex
%%% End:
