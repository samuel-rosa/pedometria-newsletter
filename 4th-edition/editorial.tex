\title{Editorial}
\maketitle

Pinturas rupestres, sinais de fumaça, sons, pergaminhos, jornais, revistas, rádio, televisão, sites, blogs, mídias sociais, MSN, Skype, Twitter, Viber, WhatsApp dentre outros. Resumidamente, essa é a evolução dos canais de comunicação e meios de divulgação utilizados pelo homem nos últimos 42 mil anos. 

Para acompanhar e estimular a reflexão daqueles envolvidos na produção e divulgação da ciência, a \pedometria{} também vem evoluindo. Avaliação das estatísticas de acesso para desenvolvimento de indicadores, inclusão nas mídias sociais (\href{https://www.facebook.com/pedometrianews?fref=ts}{Facebook}, \href{https://www.youtube.com/channel/UCWf449Ctv5X5cZqYXmAojUg}{YouTube}), e criação do espaço \pedoartemetria, são algumas das novidades da \pedometria{} a partir dessa edição. Auxiliando na evolução e aprimoramento da divulgação e popularização da ciência do solo, especialmente a pedometria, a equipe da \pedometria{} aumentou de tamanho, contando agora com a colaboração de \href{http://lattes.cnpq.br/4230318899238716}{André Geraldo de Lima Moraes} e \href{http://lattes.cnpq.br/8836609544198735}{Diego Silva Siqueira} em seu corpo editorial.

A inserção da \pedometria{} nesses novos canais de comunicação tem por objetivo propiciar mais espaços para a troca de conhecimentos e a proposição de novos modos de pensar a divulgação do tema solo e pedometria na sociedade. Dessa forma, acreditamos que possa ser gerado um habitat inovativo, onde diferentes pessoas, ideias e pontos de vista possam misturar-se, tornando-se cada vez mais proativos e abertos, gerando novos conhecimentos e pontos de vista.

É com esse espírito de construir um habitat para novas ideias e difundir os conhecimentos que a quarta edição de \pedometria{} traz José Alexandre Demattê e Marilusa Lacerda, apresentando o projeto Biblioteca Espectral de Solos do Brasil (BESB), e as perspectivas para o uso de espectroscopia na quantificação de atributos e na classificação de solos. Eles também convidam a comunidade científica a participar do Projeto por meio da doação de amostras de solos e/ou curvas espectrais.

Trazendo experiências, pontos de vista e opiniões, o jovem estudante de Engenharia Ambiental Yuri Gelsleichter fala sobre suas aventuras como cientista do solo sem fronteiras, e em entrevista, Paulo Ivonir Gubiani fala sobre os estudos de modelagem em ciência do solo, suas maiores dificuldades e perspectivas.

Para refletir, Otávio Antônio de Camargo, do Instituto Agronômico de Campinas (IAC), provoca uma série de reflexões sobre o contexto atual da ciência do solo e a necessidade de novas posturas, e Vidal Barrón, da Universidade de Córdoba, mostra a importância de outras maneiras de difundir a ciência.

A comissão editorial da \pedometria{} deseja a você ótima leitura.

%%% Local Variables: 
%%% mode: latex
%%% TeX-master: 4th-edition.tex
%%% End: 