\title{Instruções aos autores}
\maketitle
\newcommand{\SBCS}{\href{http://www.sbcs.org.br/}{SBCS}}
\newcommand{\Nature}{\href{http://www.nature.com/nature/authors/gta/}{Nature}}
\newcommand{\Science}{\href{http://www.sciencemag.org/about/authors}{Science}}
\newcommand{\CreativeCommons}{\href{http://creativecommons.org/licenses/by-sa/2.0/}{CC-BY-SA}}
\newcommand{\Kile}{\href{http://kile.sourceforge.net/}{Kile}}
\newcommand{\MiKTeX}{\href{http://miktex.org/}{MiKTeX}}
\newcommand{\aquiLaTeX}{\href{https://docs.google.com/document/d/1F3IXzWNCeUrwKeFxA4iHS3aw1nforK1IqB126xbawvA/edit?usp=sharing}{aqui}}
\newcommand{\aquiWord}{\href{https://docs.google.com/document/d/1pyK03RBDPhOMrTVfvj9NcEfjGWeswqYbYDqEKq83drQ/edit?usp=sharing}{aqui}}
\newcommand{\Geoderma}{\href{http://www.elsevier.com/author-schemas/latex-instructions}{Geoderma}}
\newcommand{\EJSS}{\href{http://onlinelibrary.wiley.com/journal/10.1111/\%28ISSN\%291365-2389/homepage/ForAuthors.html}{European Journal of Soil Science}}

\subsection{Escopo e política}

\pedometria{} é a \textit{Newsletter} (boletim técnico-científico) da Comissão de Pedometria da Sociedade Brasileira de Ciência do Solo (\SBCS). Assim sendo, dedica-se à publicação de artigos relacionados à aplicação de métodos matemáticos e estatísticos para o estudo da gênese e distribuição dos solos, o que abrange desde discussões conceituais até a aplicação prática de sensores e modelos. Especificamente, os seguintes tópicos costumam ser cobertos:

\begin{itemize}
 \item explicitação de conceitos pedométricos;
 \item entrevista com pedometrista;
 \item opinião sobre tema pedométrico relevante;
 \item descrição de equipamentos e sensores remotos;
 \item descrição de softwares e suas funcionalidades;
 \item eventos técnico-científicos;
 \item expressão artística em pedometria;
 \item novas publicações científicas em pedometria.
\end{itemize}

Os artigos submetidos para publicação em \pedometria{} NÃO DEVEM ser formais como aqueles geralmente submetidos para revistas científicas tradicionais da área de pedometria. DEVE-SE usar linguagem CLARA e INFORMAL, preferencialmente na VOZ ATIVA. O uso da voz ativa é uma recomendação para publicação em ambas as revistas \Nature{} e \Science:

\begin{description}
 \item \textit{Nature} ``Nature journals like authors to write in the active voice (`we performed the experiment...') as experience has shown that readers find concepts and results to be conveyed more clearly if written directly.''
 \item \textit{Science} ``Use active voice when suitable, particularly when necessary for correct syntax (e.g., `To address this possibility, we constructed a lZap library ...,' not `To address this possibility, a lZap library was constructed...').''
\end{description}

Conceitos pedométricos devem ser introduzidos com SENSIBILIDADE, tendo em mente que os leitores podem os desconhecer e/ou não possuir base conceitual suficiente para compreendê-los em uma única leitura. A estrutura do artigo deve ser concebida da mesma maneira que o fazemos para contar uma história a um amigo ou familiar, a fim de cativar o leitor e envolvê-lo no emocionante universo da pedometria. Caso a comissão editorial entenda que a compreensão do texto exija aprofundado conhecimento prévio do leitor, os autores serão solicitados a torná-lo mais simples e amigável.

O escopo e política de \pedometria{} coloca-a como veículo de divulgação e desmistificação da pedometria no Brasil. Trata-se de uma publicação com três edições anuais que permite aos pesquisadores brasileiros divulgar suas pesquisas pedométricas e, sobretudo, conhecerem uns aos outros. Isso é importante porque, assim como a própria pedometria, a maioria dos pedometristas também são bastante jovens, muitos dos quais ainda estão desenvolvendo seus estudos de mestrado e/ou doutorado. Por este motivo, \pedometria{} é distribuída gratuitamente via Internet, estando registrada sob a licença Creative Commons Atribuição-Compartilha Igual 3.0 Não Adaptada (\CreativeCommons).\\
\\
\begin{figure}[h!]
 \centering
 \includegraphics[width=0.8\textwidth]{figuras/cc-by-sa}
 \caption{Licença Creative Commons Atribuição-Compartilha Igual 3.0 Não Adaptada.}
\end{figure}

\subsection{Estrutura do artigo}

A estrutura do artigo é inteiramente definida pelo(s) autor(es). Sugere-se que sejam utilizadas subdivisões em até dois níveis de profundidade (seções e subseções), não necessariamente definidas como em artigos científicos tradicionais. Os artigos podem ter, além do título, um subtítulo. Resumo e palavras-chave não são utilizados.

\subsubsection{Figuras e tabelas}

Figuras e tabelas são recomendadas, sendo uma foto do(s) autor(es) obrigatória. Figuras devem ser preparadas no formato PNG, com resolução mínima de 300 dpi.

\subsubsection{Referências bibliográficas}

Referências bibliográficas não são mandatórias, sobretudo no caso de artigos apresentando a opinião do autor sobre um tema pedométrico. Caso citações sejam feitas ao longo do texto, a lista de referências deve ser organizada na ordem em que as citações são feitas no texto. Importante notar que o estilo de citação numérica sobrescrita da \textit{Nature} é utilizado.

A lista de referências bibliográficas deve conter até cinquenta itens. O conteúdo da lista de referências deve ser econômico, incluindo apenas informações fundamentais para a sua identificação. Artigos, trabalhos em anais de eventos ou em coleções, e trabalhos publicados em veículos periódicos similares, devem ser formatados da seguinte maneira:

\vspace{0.5cm}
\noindent{A.B. McBratney, M.L. Mendonça-Santos, B. Minasny (2003) On digital soil mapping. \textit{Geoderma}, 117:3-52. \doi{10.1016/S0016-7061(03)00223-4}}
\vspace{0.5cm}

\noindent{onde \textit{link} refere-se ao endereço na Internet onde o referido documento pode ser encontrado. No caso de artigos científicos recomenda-se utilizar o Digital Object Identifier (\href{http://www.doi.org/}{DOI}). Livros, teses, dissertações, relatórios e outros meios de publicação não periódicos devem ser formatados da seguinte maneira:}

\vspace{0.5cm}
\noindent{H. Jenny (1994) \textit{Factors of soil formation - a system of quantitative pedology}. Toronto: Dover Publications.}
\vspace{0.5cm}

\noindent{Caso as publicações não periódicas estejam disponíveis na Internet, um link deve ser adicionado assim como para as publicações periódicas.}

\subsection{Preparo do artigo}

Os artigos podem ser preparados utilizando processadores de texto tradicionais, como por exemplo o MS Office Word (*.doc) e o LibreOffice Writer (*.odt), ou a linguagem de marcação \LaTeX() (*.tex). Preferência deve ser dada ao formato \LaTeX() sempre que possível a fim de facilitar o trabalho da equipe editorial. Documentos no formato PDF não são aceitos.

\subsubsection{Word ou Writer}

Um modelo para preparo do artigo usando processadores de texto tradicionais pode ser acessado clicando \aquiWord.

\subsubsection{\LaTeX}

\href{http://pt.wikipedia.org/wiki/Latex}{\LaTeX} é uma \href{http://pt.wikipedia.org/wiki/Linguagem_de_marca\%C3\%A7\%C3\%A3o}{linguagem de marcação} amplamente utilizada para a produção de textos matemáticos e científicos devido à sua alta qualidade tipográfica. Utilizar uma linguagem de marcação para elaboração de textos constitui no uso de comandos escritos para definir a formatação do documento, ao contrário dos editores tradicionais que oferecem abas e caixas de diálogo para clicar e definir parâmetros de formatação. Na prática, ao produzir um texto em \LaTeX, o autor não vê o produto final formatado na tela do computador imediatamente, mas apenas o texto e os comandos de formatação. O objetivo é distanciar o autor o máximo possível da apresentação visual do artigo, ou seja, ao invés de trabalhar com ideias visuais, o autor é encorajado a trabalhar com conceitos lógicos independentes da apresentação final do artigo. Além disso, linguagens de marcação como o \LaTeX{} permitem agilizar a confecção do documento final (PDF) e reproduzir o conteúdo em outros formatos para apresentação em meio digital como o html. Atualmente, importantes revistas científicas da área de pedometria adotam o \LaTeX{} para a confecção de seus artigos, como por exemplo \Geoderma{} e \EJSS.

Apesar de diferente dos processadores de texto tradicionais, usar \LaTeX{} é bastante fácil, sendo necessário apenas compreender a sua lógica de funcionamento. Para isso é bom dar uma olhada no \href{http://www.stdout.org/~winston/latex/}{Manual Rápido \LaTeX{}} para conhecer alguns comandos básicos. O preparo dos artigos usando \LaTeX{} pode ser feito usando softwares como Bloco de Notas, WordPad, e Gedit. Entretanto, é mais aconselhado usar um software específico para a edição de documentos em \LaTeX, como o \Kile{} e o \MiKTeX. Tais softwares auxiliam a inserção dos comandos necessários para a formatação do texto sem a necessidade de memorizá-los.

Um modelo para preparo do artigo usando \LaTeX{} pode ser acessado clicando \aquiLaTeX.

\subsection{Submissão}

Qualquer pessoa pode submeter um artigo para publicação em \pedometria{} sem qualquer custo. Quando o artigo estiver pronto, adicione todos os arquivos utilizados (inclusive as figuras originais) à uma pasta comprimida e envie para a comissão editorial usando o endereço de e-mail \email{pedometria.news@gmail.com}.

\subsection{Comissão editorial}

Alessandro Samuel-Rosa\\
\textit{ISRIC - World Soil Information}\\
\textit{Wageningen University and Research Center}\\
\email{alessandro.rosa@wur.nl}\\
\\
André Geraldo de Lima Moraes\\
\textit{Curso de Pós-Graduação em Agronomia-Ciência do Solo}\\
\textit{Universidade Federal Rural do Rio de Janeiro}\\
\email{andrehmuz@hotmail.com}\\
\\
Diego Silva Siqueira\\
\textit{Grupo de Pesquisa Caracterização do Solo para fins de Manejo Específico}\\
\textit{Universidade Estadual Paulista - Campus Jaboticabal}\\
\email{diego\_silvasiqueira@yahoo.com.br}\\
\\
Jean Michel Moura-Bueno\\
\textit{Programa de Pós-Graduação em Ciência do Solo}\\
\textit{Universidade Federal de Santa Maria}\\
\email{bueno.jean1@gmail.com}

\subsection{Observação}

Iniciamos tratativas com a Sociedade Brasileira de Ciência do Solo para que toda a sua documentação e números publicados de \pedometria{} fiquem hospedados no \textit{website} \url{http://www.sbcs.org.br}. Infelizmente a Sociedade  Brasileira de Ciência do Solo não possui infraestrutura disponível para que isso seja possível. Enquanto isso, \pedometria{} fica temporariamente hospedada no \textit{website} \url{https://sites.google.com/site/pedometria/}, impedindo a obtenção de ISSN.

\subsection{Agradecimento}

Gostaríamos de agradecer o conselho editorial do \href{http://grass.osgeo.org/newsletter/}{GRASS News} do projeto GRASS pois, sem a sua contribuição com os arquivos *.tex e *.sty do GRASS News, \pedometria{} teria levado muito mais tempo para atingir o nível atual.
%%% Local Variables: 
%%% mode: latex
%%% TeX-master: 4rd-edition.tex
%%% End: