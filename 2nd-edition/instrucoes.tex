\title{Instruções aos autores}
\maketitle
\subsection{Escopo e política}
\textbf{\em Pedometria} é a \textit{Newsletter} (boletim técnico-científico) da Comissão de Pedometria da Sociedade Brasileira de Ciência do Solo (\href{http://www.sbcs.org.br/}{SBCS}). Assim sendo, dedica-se à publicação de trabalhos relacionados à aplicação de métodos matemáticos e estatísticos para o estudo da gênese e distribuição dos solos. Os artigos não devem ser tão científicos/profissionais como nas revistas científicas tradicionais, usando um linguagem clara e informal, em português, preferencialmente na voz ativa, e introduzindo conceitos pedométricos com sensibilidade a fim de envolver o leitor no emocionante universo da pedometria. Como o processo de revisão visa apenas garantir a qualidade das contribuições, o processo de publicação é mais rápido e eficiente. Isso contribui para cobrir mais rapidamente a lacuna de informações em pedometria na Brasil.\\
A \textit{Newsletter} coloca-se como veículo de divulgação e desmistificação da pedometria no Brasil, uma publicação com três edições anuais que permite aos pesquisadores brasileiros divulgar suas pesquisas pedométricas e, sobretudo, conhecerem uns aos outros. Isso é importante porque, assim como a própria pedometria, a maioria dos pesquisadores brasileiros dessa área também são bastante jovens, muitos dos quais ainda estão desenvolvendo seus estudos de mestrado e doutorado. Por este motivo, a \textit{Newsletter} é distribuída gratuitamente, estando registrada sob a licença Creative Commons Atribuição-Compartilha Igual 3.0 Não Adaptada (\href{http://creativecommons.org/licenses/by-sa/2.0/}{CC-BY-SA}).\\
\\
\begin{figure}[h!]
\centering
\includegraphics[scale=0.8]{figuras/cc-by-sa.png}
\caption{Atribuição-Compartilha Igual 3.0 Não Adaptada.}
\end{figure}
\noindent Qualquer pessoa interessada pode submeter um texto para publicação na \textit{Newsletter}. Basta usar os modelos disponíveis na página da \href{https://sites.google.com/site/pedometria/file-cabinet}{\textit{Newsletter}} para elaboração dos artigos e contactar o Editor-chefe por e-mail (\email{alessandrosamuel@yahoo.com.br}). A \textit{Newsletter} está aberta aos mais diversos temas relacionados à pedometria, desde discussões sobre conceitos até a aplicação de sensores e modelos matemáticos. Os seguintes tópicos são cobertos: (a) explicitação de conceitos usados em pedometria, (b) entrevista de pesquisadores brasileiros, (c) a opinião de pesquisadores sobre determinado tema relevante para a pedometria, (d) descrição de equipamentos e sensores remotos, (e) descrição de softwares e suas funcionalidades, (f) eventos técnico-científicos da área, e (g) novas publicações científicas em pedometria.
\subsection{Submissões}
Os artigos devem ser preparados em \href{http://www.latex-project.org/}{LaTeX}. Se você ainda não sabe LaTeX, não se assuste, pois LaTeX é muito fácil. Você pode dar uma olhada no \href{http://www.stdout.org/~winston/latex/}{Manual Rápido LaTeX} para conhecer alguns comandos, como por exemplo como enumerar, fazer tabelas, etc. A edição dos arquivos tex pode ser feita diretamente em processadores de texto tradicionais como o MS Word e \href{http://www.libreoffice.org/}{LibreOffice}, e até mesmo editores de blocos de notas. Para opções mais avançadas de edição de tex podem ser usados softwares específicos como o \href{http://kile.sourceforge.net/}{Kile} e o \href{http://miktex.org/}{MiKTeX}.\\
Note que, preferencialmente, os artigos não devem exceder 10 páginas. Todas as imagens, inclusive uma foto sua a ser adicionada no cabeçalho, devem ser em formato PNG, enquanto desenhos e diagramas vetoriais devem estar no formato PDF. Adicione seus arquivos à uma pasta comprimida e envie para o Editor-chefe usando o endereço de e-mail \email{alessandrosamuel@yahoo.com.br}.
\subsection{Sobre nós}
\subsubsection{Editor-chefe}
Alessandro Samuel-Rosa\\
\textit{Curso de Pós-Graduação em Agronomia-Ciência do Solo}\\
\textit{Universidade Federal Rural do Rio de Janeiro}\\
\email{alessandrosamuel@yahoo.com.br}
\subsubsection{Editor da seção de entrevistas, eventos e novas publicações}
Jean Michel Moura-Bueno\\
\textit{Programa de Pós-Graduação em Ciência do Solo}\\
\textit{Universidade Federal de Santa Maria}\\
\email{bueno.jean1@gmail.com}
\subsection{Observação}
No momento estamos em tratativas com a SBCS para que toda a sua documentação e números publicados fiquem hospedados no \textit{website} \url{http://www.sbcs.org.br}. Enquanto isso, a \textit{Newsletter} fica temporariamente hospedada no \textit{website} \url{https://sites.google.com/site/pedometria/}.\\
Gostaríamos de agradecer o conselho editorial do \href{http://grass.osgeo.org/newsletter/}{GRASS News} do projeto GRASS pois, sem a sua contribuição com os arquivos *.tex e *.sty do GRASS News, nossa \textit{Newsletter} teria levado muito mais tempo para atingir o nível atual.
%%% Local Variables: 
%%% mode: latex
%%% TeX-master: documento-principal.tex
%%% End: