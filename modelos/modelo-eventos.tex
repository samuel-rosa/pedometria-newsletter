%%%%%%%%%%%%%%%%%%%%%%%%%%%%%%%%%%%%%%%%%%%%%%%%%%%%%%%%%%%%%%%%%%%%%%
% Modelo de ARTIGO DE EVENTOS da newsletter da
% Comissão de Pedometria da Sociedade Brasileira de Ciência do Solo
%
% Esse modelo deve ser utilizado para a construção de ARTIGOS DE
% EVENTOS.
% 
% Recomendações gerais:
% A edição desse documento deve ser feita utilizando um editor LaTeX
% como o RStudio.
% Comentários precedidos por '%' não precisam ser deletados pois não
% são reconhecidos durante a copilação do documento.
% Elementos como '\subsection{}' constituem comandos e não devem
% ser alterados, exceto o conteúdo entre parênteses.
% Ex.: '\subsection{Título do evento}' = '\subsection{Congresso de Solos}'
%
% Altere o nome do arquivo para 'eventos.tex'.
% 
% Não use figuras.
% 
% Não inclua seu nome e/ou sua foto como autor do artigo.
%
% Use 'subsection' e 'subsubsection'.
%
% Language: Latex
%%%%%%%%%%%%%%%%%%%%%%%%%%%%%%%%%%%%%%%%%%%%%%%%%%%%%%%%%%%%%%%%%%%%%%

% Não altere as duas linhas seguintes.
\title{Eventos}
\maketitle

% Aqui vão as informações básicas sobre o evento. Repita para cada evento.
\subsection{Título do evento}
\subsubsection{Cidade, país, data}

% O texto descritivo do evento vai aqui.
O evento será organizado pelo Instituto de Ciência do Gato Xadrez. As inscrições iniciam em 31/02 e o prazo para a submissão de trabalhos é 01/04. Maiores informações podem ser encontradas no site \url{http://gato.xadrez.miau/} ou escrevendo para \email{gato@xadrez.miau}.


%%% Local Variables: 
%%% mode: latex
%%% TeX-master: documento-principal.tex
%%% End: